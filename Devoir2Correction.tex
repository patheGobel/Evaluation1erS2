\documentclass[12pt,a4paper]{article}
\usepackage{amsmath,amssymb,mathrsfs,tikz,times,pifont}
\usepackage{enumitem}
%+++++++++++++++++++++++++++++++++++++++++++++++++
\usepackage{stmaryrd}
\usepackage{graphicx} % Pour l'insertion d'images
\usepackage{float}    % Pour contrôler précisément le placement
\usepackage[utf8]{inputenc}

\usepackage[french]{babel}
\usepackage[T1]{fontenc}
\usepackage{hyperref}
\usepackage{verbatim}

\usepackage{color, soul}

\usepackage{pgfplots}
\pgfplotsset{compat=1.15}
\usepackage{mathrsfs}

\usepackage{amsmath}
\usepackage{amsfonts}
\usepackage{amssymb}
\usepackage{tkz-tab}
%+++++++++++++++++++++++++++++++++++++++++++++++++
\newcommand\circitem[1]{%
\tikz[baseline=(char.base)]{
\node[circle,draw=gray, fill=red!55,
minimum size=1.2em,inner sep=0] (char) {#1};}}
\newcommand\boxitem[1]{%
\tikz[baseline=(char.base)]{
\node[fill=cyan,
minimum size=1.2em,inner sep=0] (char) {#1};}}
\setlist[enumerate,1]{label=\protect\circitem{\arabic*}}
\setlist[enumerate,2]{label=\protect\boxitem{\alph*}}
%%%::::::by chnini ameur :::::::%%%
\everymath{\displaystyle}
\usepackage[left=1cm,right=1cm,top=1cm,bottom=1.7cm]{geometry}
\usepackage{array,multirow}
\usepackage[most]{tcolorbox}
\usepackage{varwidth}
\tcbuselibrary{skins,hooks}
\usetikzlibrary{patterns}
%%%::::::by chnini ameur :::::::%%%
\newtcolorbox{exa}[2][]{enhanced,breakable,before skip=2mm,after skip=5mm,
colback=yellow!20!white,colframe=black!20!blue,boxrule=0.5mm,
attach boxed title to top left ={xshift=0.6cm,yshift*=1mm-\tcboxedtitleheight},
fonttitle=\bfseries,
title={#2},#1,
% varwidth boxed title*=-3cm,
boxed title style={frame code={
\path[fill=tcbcolback!30!black]
([yshift=-1mm,xshift=-1mm]frame.north west)
arc[start angle=0,end angle=180,radius=1mm]
([yshift=-1mm,xshift=1mm]frame.north east)
arc[start angle=180,end angle=0,radius=1mm];
\path[left color=tcbcolback!60!black,right color = tcbcolback!60!black,
middle color = tcbcolback!80!black]
([xshift=-2mm]frame.north west) -- ([xshift=2mm]frame.north east)
[rounded corners=1mm]-- ([xshift=1mm,yshift=-1mm]frame.north east)
-- (frame.south east) -- (frame.south west)
-- ([xshift=-1mm,yshift=-1mm]frame.north west)
[sharp corners]-- cycle;
},interior engine=empty,
},interior style={top color=yellow!5}}
%%%%%%%%%%%%%%%%%%%%%%%

\usepackage{fancyhdr}
\usepackage{eso-pic}         % Pour ajouter des éléments en arrière-plan
% Commande pour ajouter du texte en arrière-plan
\AddToShipoutPicture{
    \AtTextCenter{%
        \makebox[0pt]{\rotatebox{80}{\textcolor[gray]{0.7}{\fontsize{5cm}{5cm}\selectfont PGB}}}
    }
}
\usepackage{lastpage}
\fancyhf{}
\pagestyle{fancy}
\renewcommand{\footrulewidth}{1pt}
\renewcommand{\headrulewidth}{0pt}
\renewcommand{\footruleskip}{10pt}
\fancyfoot[R]{
\color{blue}\ding{45}\ \textbf{2024}
}
\fancyfoot[L]{
\color{blue}\ding{45}\ \textbf{Prof:M. BA}
}
\cfoot{\bf
\thepage /
\pageref{LastPage}}
\begin{document}
\renewcommand{\arraystretch}{1.5}
\renewcommand{\arrayrulewidth}{1.2pt}
% Nouvelle commande pour entourer un nombre avec un cercle
\newcommand{\encercle}[1]{\tikz[baseline=(X.base)] \node (X) [draw, circle, inner sep=1pt] {#1};}
\begin{tikzpicture}[overlay,remember picture]
\node[draw=blue,line width=1.2pt,fill=purple,text=blue,inner sep=3mm,rounded corners,pattern=dots]at ([yshift=-2.5cm]current page.north) {\begingroup\setlength{\fboxsep}{0pt}\colorbox{white}{\begin{tabular}{|*1{>{\centering \arraybackslash}p{0.28\textwidth}} |*2{>{\centering \arraybackslash}p{0.2\textwidth}|} *1{>{\centering \arraybackslash}p{0.19\textwidth}|} }
\hline
\multicolumn{3}{|c|}{$\diamond$$\diamond$$\diamond$\ \textbf{Lycée de Dindéfélo}\ $\diamond$$\diamond$$\diamond$ }& \textbf{A.S. : 2024/2025} \\ \hline
\textbf{Matière: Mathématiques}& \textbf{Niveau : 1}\textbf{$^{er}$S2} &\textbf{Date: 21/01/2025} & \textbf{Durée : 4 heures} \\ \hline
\multicolumn{4}{|c|}{\parbox[c]{10cm}{\begin{center}
\textbf{{\Large\sffamily Devoir n$ ^{\circ} $ 2 Du 1$ ^\text{\bf er} $ Semestre}}
\end{center}}} \\ \hline
\end{tabular}}\endgroup};
\end{tikzpicture}
\vspace{3cm}

\section*{\underline{Exercice 1 :} 3 pts [\textit{ Déjà corrigé en classe par moi-même} ]}
Soit \( ABC \) un triangle tel que \( AB = 4\,\text{cm} \) et \( BC = 5\,\text{cm} \).

On donne \( I = \text{bar} \{(A,3), (B,-2)\} \) et \( J = \text{bar} \{(B,-2), (C,3)\} \), soit \( G \) le point défini par :
\[
3\overrightarrow{GA} - 2\overrightarrow{GB} + 3\overrightarrow{GC} = \overrightarrow{0}.
\]

\begin{enumerate}
    \item Montrer que \( G = \text{bar} \{(I,-1), (C,3)\} \). \hfill \textbf{0,75 pt}
    \item Déterminer et construire l'ensemble des points \( P \) du plan tels que :
    \begin{enumerate}
        \item \( \|3\overrightarrow{MA} - 2\overrightarrow{MB} + 3\overrightarrow{MC}\| = AB \),\hfill \textbf{0,75 pt}
        \item \( \|18\overrightarrow{MA} - 12\overrightarrow{MB}\| = \|\overrightarrow{MA} - \overrightarrow{MB}\| \),\hfill \textbf{0,75 pt}
        \item \( \|3\overrightarrow{MA} - 9\overrightarrow{MB}\| = \| -2\overrightarrow{MB} + 3\overrightarrow{MC}\| \).\hfill \textbf{0,75 pt}
    \end{enumerate}
\end{enumerate}

\section*{\underline{Exercice 2 :} 7 points [\textit{ Exercice 2 du devoir 1 déjà corrigé en classe } ]}

On donne \( P(x) = 2x^3 + ax^2 + bx - 6 \) où \( a \) et \( b \) sont des réels.
\begin{enumerate}
    \item Trouver les réels \( a \) et \( b \) pour que le polynôme \( P(x) \) soit divisible par \( x^2 - x - 2 \).
    \item En déduire une factorisation de \( P(x) \).
    \item Résoudre dans \( \mathbb{R} \) l'équation \( P(x) = 0 \).
    \item En déduire les solutions de l'équation \( P(x^2 - 2) = 0 \).
    \item Résoudre dans \( \mathbb{R} \) l'inéquation \( P(x) \leq 0 \).
    \item En déduire les solutions de l'inéquation \( P(3 - 2x) \leq 0 \).
    \item Résoudre dans \( \mathbb{R} \) l'inéquation \( \frac{2x^3 + x^2 - 7x - 6}{-x + 1} > 0 \).
\end{enumerate}
\section*{\underline{\textcolor{green}{Correction Exercice 2 :} 7 points}}

On donne \( P(x) = 2x^3 + ax^2 + bx - 6 \) où \( a \) et \( b \) des réels.

\begin{enumerate}
    \item Déterminons \( a \) et \( b \) pour que \( P(x) \) soit divisible par \( x^2 - x - 2 \).

    \item[\( \bullet \)] Si \( x^2 - x - 2 \) divise \( P(x) \), alors les racines de \( x^2 - x - 2 = 0 \) sont aussi racines de \( P(x) = 0 \).

Or les racines de \( x^2 - x - 2 = 0 \) sont \( x_1 = -1 \) et \( x_2 = 2 \).

\[
\begin{cases}
P(-1) = 0 \\
P(2) = 0
\end{cases}\implies 
\begin{cases}
2(-1)^3 + a(-1)^2 + b(-1) - 6 = 0 \\
2(2)^3 + a(2)^2 + b(2) - 6 = 0 
\end{cases}\implies 
\begin{cases}
-2 + a - b - 6 = 0 \\
16 + 4a + 2b - 6 = 0
\end{cases}
\]
\[
\begin{cases}
a - b  = 8 \\
4a + 2b = -10
\end{cases}\implies 
\begin{cases}
a - b  = 8 \\
2a + b = -5
\end{cases}
\]

\[
\underline{
\begin{cases}
a - b  = 8 \\
2a + b = -5
\end{cases}
}\\
\]
\begin{center}
$3a=3 \implies a=1$\\
\text{Remplaçons $a$ dans  $a - b  = 8 \implies b=-7$ }
Donc \( P(x) = 2x^3 + x^2 - 7x - 6 \)
\end{center}
\item Une factorisation de \( P(x) \).
\[
\begin{array}{|c|c|c|c|c|}
\hline
 & 2 & 1 & -7 & -6  \\ 
\hline
-1 & \times & -2 & 1 & 6 \\ 
\hline
 & 2 & -1 & -6 & 0  \\
\hline
2 & \times & 4 & 6 &  \\ 
\hline
& 2 & 3 & 0 &   \\
\hline
\end{array}
\]
donc \( P(x) = (x+1)(x-2)(2x+3) \)

\item Résolvons \( P(x) = 0 \)
\[
P(x) = 0 \implies (x+1)(x-2)(2x+3) = 0
\]
  
\[
x = -1 \quad \text{ou} \quad x = 2 \quad \text{ou} \quad x = -\frac{3}{2}
\]

Ainsi,  
\[
\textcolor{green}{\boxed{S_{\mathbb{R}} = \left\{ -1, 2, -\frac{3}{2} \right\}}}
\]

\item Déduisons-en les solutions de \( P(x^2 - 2) = 0 \)
 \[ P(x) = (x+1)(x-2)(2x+3) \]
\[
P(x^2 - 2) = (x^2 - 2 + 1)(x^2 - 2 - 2)(2(x^2 - 2) + 3)
\]

\[
P(x^2 - 2) = 0 \implies (x^2 - 1)(x^2-4)(2x^2 - 1) = 0
\]
 
\[
x = -1 \quad \text{ou} \quad x = 1 \quad \text{ ou } \quad x=-2 \quad \text{ ou } \quad x=2\quad x = \sqrt{\frac{1}{2}} \quad \text{ou} \quad x = -\sqrt{\frac{1}{2}}
\]

Ainsi,  
\[
\textcolor{green}{\boxed{S_{\mathbb{R}} = \left\{ -1, 1, -2,2, \sqrt{\frac{1}{2}}, -\sqrt{\frac{1}{2}} \right\}}} 
\]

\item  Résolvons \( P(x) \leq 0 \)

\begin{center}
\begin{tikzpicture}
    % Création du tableau
    \tkzTabInit{$x$ / 1 , $x+1$ / 1 , $x-2$ / 1 , $2x+3$ / 1 , $P(x)$ / 1}
      { $-\infty$, $-\frac{3}{2}$, $-1$, $2$, $+\infty$ }

    % Ligne x+1
    \tkzTabLine{ , - ,  , - , z ,+, ,+, }
    % Ligne x-2
    \tkzTabLine{ , - ,  , - ,  ,-, z,+, }
    % Ligne 2x+3
    \tkzTabLine{ , - , z , + ,  ,+, ,+, }
    % Ligne P(x)
    \tkzTabLine{ , - , z , + , z ,-, z,+, }
\end{tikzpicture}
\end{center}
\[
\textcolor{green}{\boxed{S = \left] -\infty ; -\frac{3}{2}\right] \cup \left[ -1;2\right] }} 
\]
\item Déduisons-en les positions de \( P(3-2x) \leq 0 \)

\[
P(3-2x) \leq 0 \implies 
\left( 3 - 2x + 1 \right)\left( 3 - 2x - 2 \right)\left[ 2\left( 3 - 2x \right) + 3 \right] \leq 0
\]

\[
\left( -2x + 4 \right)\left( -2x + 1 \right)\left( -4x + 9 \right) \leq 0
\]
\begin{center}
\begin{tikzpicture}
    % Initialisation du tableau avec largeur ajustée pour la première colonne
    \tkzTabInit[lgt=3]{$x$ / 1.5, $-2x+4$ / 1, $-2x+1$ / 1, $-4x+9$ / 1, $P(3-2x)$ / 1}
      { $-\infty$, $\frac{1}{2}$, $2$, $\frac{9}{4}$, $+\infty$ }

    % Ligne -2x+4
    \tkzTabLine{ , + , + , + , z , -,-,-, }
    % Ligne -2x+1
    \tkzTabLine{ , + , z , - , - , -,-,-, }
    % Ligne -4x+9
    \tkzTabLine{ , + , + , z , - , -,z,-, }
    % Ligne P(3-2x)
    \tkzTabLine{ , + , z , - , z , +,z,-, }
\end{tikzpicture}
\end{center}
\[
\textcolor{green}{\boxed{S = \left[ \frac{1}{2} ; 2\right] \cup \left[ \frac{9}{4};+\infty\right] }} 
\]
\item Résolvons dans \( \mathbb{R} \) : \( \frac{P(x)}{-x+1} > 0 \)
\begin{center}
\begin{tikzpicture}
    % Initialisation du tableau avec une largeur ajustée pour la première colonne
    \tkzTabInit[lgt=2]{$x$ / 1.5, $P(x)$ / 1, $-x+1$ / 1, $\frac{P(x)}{-x+1}$ / 1}
      { $-\infty$, $-\frac{3}{2}$, $-1$, $1$, $2$, $+\infty$ }

    % Ligne P(x)
    \tkzTabLine{ , - , z , + , z ,-,t ,- ,z,+ }
    % Ligne -x+1
    \tkzTabLine{ , + , t , + , t , +, z , -, t , - }
    % Ligne P(x)/(-x+1)
    \tkzTabLine{ , - , z , + , z , - , d ,+ ,z,- }
\end{tikzpicture}
\end{center}

\[
\textcolor{green}{\boxed{S = \left]  -\frac{3}{2}, -1\right[   \cup \left] 1, 2\right[  }} 
\]
\end{enumerate}
\section*{\underline{Exercice 3 :} 8 pts 	[\textit{Exercice 3 du devoir 1 déjà corrigé en classe} ]}

\begin{enumerate}
    \item Résoudre dans $\mathbb{R}$ les équations suivantes.
    \begin{enumerate}
        \item $\sqrt{-x^2 + 5x + 9} = \sqrt{x - 3}$
        \item $\sqrt{x^2 - x - 2} = x + 1$
    \end{enumerate}
    
    \item Résoudre dans $\mathbb{R}$ les inéquations suivantes.
    \begin{enumerate}
        \item $\sqrt{(x + 3)(x - 1)} \leq x - 3$
        \item $\sqrt{-5x + 1} \geq \sqrt{x^2 - 3x + 2}$
        \item $\sqrt{x^{2}-x-2} > x-1 $
        \item $x^4 + x^2 - 12 \leq 0$
    \end{enumerate}
    
    \item Résoudre dans $\mathbb{R}^2$ le système suivant
    \[
    \begin{cases}
        x^3 + y^3 = 9 \\
        \frac{x}{y} + \frac{y}{x} = \frac{5}{2}
    \end{cases}
    \]
    
    \item Résoudre dans $\mathbb{R}^3$ par la méthode du Pivot de Gauss le système
    \[
    \begin{cases}
        x - 2y + z = 6 \\
        -2x + y - z = -6 \\
        3x - y - 2z = -2
    \end{cases}
    \]
\end{enumerate}

\section*{\underline{\textcolor{green}{Correction Exercice 3 :} 8 points}}
\begin{enumerate}
    \item Résolvons dans $\mathbb{R}$ les équations suivantes.
    \begin{enumerate}
        \item $\sqrt{-x^2 + 5x + 9} = \sqrt{x - 3}$
        
\[
\sqrt{-x^2 + 5x + 9} = \sqrt{x-3}\implies
\begin{cases}
x - 3 \geq 0 \quad (1) \\
-x^2 + 5x + 9 =x-3 \quad (2)
\end{cases}
\]

\textbf{Domaine de validité : \( D_v \)}

(1) \( x - 3 > 0 \)  \\
   \(
   x \geq 3
   \)\\
   Donc, \( x \in [3, +\infty[ \).

\textbf{Résolution}
 
\(
-x^2 + 5x + 9 = x - 3
\)
 
\(
x^2 - 4x - 12 = 0
\)
\(
\Delta' = 4 + 12 \implies \Delta = 16
\)

Les solutions sont :  
\(
x_1 = 2 - 4 \quad \text{et} \quad x_2 = 2 + 4
\)
\(
x_1 = -2 \quad \text{et} \quad x_2 = 6
\)

\noindent Vérification des solutions dans le domaine de validité \( D_v \) :  
\(
-2 \notin D_v \quad \text{et} \quad 6 \in D_v
\)

\(
\textcolor{green}{\boxed{S = \{ 6 \}  }} 
\)
\item $ \sqrt{x^2 - x - 2} = x + 1$
\[
\sqrt{x^2 - x - 2} = x + 1\implies
\begin{cases}
x + 1 \geq 0 \quad (1) \\
x^2 - x - 2 \geq 0 \quad (2)
\end{cases}
\]
\textbf{Domaine de validité : \( D_v \)}

\( x + 1 > 0 \implies x > -1 \) \\  
\( x \in ]-1, +\infty[ \quad \text{donc} \quad D_v = ]-1, +\infty[ \)

\textbf{Résolution}

\( x^2 - x - 2 = (x+1)^2 \)

\( x^2 - x - 2 = x^2 + 2x + 1 \)

\( -3x - 3 = 0 \implies x = -1 \)

\noindent Vérification : \( -1 \in D_v \).  
 
\(
\textcolor{green}{\boxed{S = \{ -1 \}  }} 
\)
    \end{enumerate} 
\item Résolvons dans \( \mathbb{R} \)

\begin{enumerate}
\item \(\sqrt{(x+3)(x-1)} \leq x-3\)

\[
\sqrt{(x+3)(x-1)} \leq x-3 \iff 
\begin{cases}
x-3 \geq 0 \quad (1) \\
(x+3)(x-1) \geq 0 \quad (2) \\
(x+3)(x-1) \leq (x-3)^2 \quad (3)
\end{cases}
\]
 
(1) \((x+3)(x-1) \leq (x-3)^2\)

\(
x^2 + 2x - 3 \leq x^2 - 6x + 9
\)

\(
8x - 12 \leq 0
\)

\(
2x - 3 \leq 0
\)

\noindent donc le système d'inéquation devient :  
\[
\begin{cases}
x-3 \geq 0 \quad (1) \\
x^2 + 2x - 3 \geq 0 \quad (2) \\
2x - 3 \leq 0 \quad (3)
\end{cases}
\]
 
\[\textbf{Posons}
\begin{cases}
x-3 = 0 \quad (1) \\
x^2 + 2x - 3 = 0 \quad (2) \\
2x - 3 = 0 \quad (3)
\end{cases}
\]
 
(1) \( x - 3 = 0 \implies x = 3 \)

(2) \( x^2 + 2x - 3 = 0 \implies x = -3 \quad \text{et} \quad x = 1 \)

(2)  \( 2x - 3 = 0 \implies x = \frac{3}{2} \)


\begin{flushleft}
\begin{tikzpicture}
    \tkzTabInit[lgt=2.5]{$x$ / 1, $x-3$ / 1, $x^2+2x-3$ / 0.6, $2x-3$ / 0.6}
      {$-\infty$, $-3$, $1$, $\frac{3}{2}$, $3$, $+\infty$}

    % Ligne x-3
    \tkzTabLine{ , - , t , - , t , -, t ,-,z,+ }
    % Ligne x^2+2x-3
    \tkzTabLine{ , + , z , - , z , +, t ,+,+,+ }
    % Ligne 2x-3
    \tkzTabLine{ , - , t , - , t , -,z,+,+,+ }
\end{tikzpicture}
\end{flushleft}
\begin{center}
\(
\textcolor{green}{\boxed{S = \emptyset }} 
\)
\end{center}

\item $\sqrt{-5x+1} > \sqrt{x^2 - 3x + 8}$
\[
\sqrt{-5x+1} > \sqrt{x^2 - 3x + 8} \implies 
\begin{cases}
-5x+1 \geq 0  \quad \textbf{(1)}\\
x^2 - 3x + 8 \geq 0 \quad \textbf{(2)}\\
-5x+1 > x^2 - 3x + 8 \quad \textbf{(3)}
\end{cases}\implies
\begin{cases}
-5x+1 \geq 0  \quad \textbf{(1)}\\
x^2 - 3x + 2 \geq 0 \quad \textbf{(2)}\\
x^2 + 2x + 1 \leq 0 \textbf{(3)}
\end{cases}
\]

\[\text{Posons }
\begin{cases}
-5x+1 \geq 0  \quad \textbf{(1)}\\
x^2 - 3x + 2 \geq 0 \quad \textbf{(2)}\\
x^2 + 2x + 1 \leq 0 \textbf{(3)}
\end{cases}
\]

\textbf{(1)} \( -5x+1 = 0 \implies x = \frac{1}{5} \)

\textbf{(2)} \( x^2 - 3x + 2 \geq 0 \implies x=1 \text{ ou } x=2\)

\textbf{(3)} \( x^2 + 2x + 1 = 0 \implies x=-1\)

\begin{flushleft}
\begin{tikzpicture}
    % Initialisation du tableau avec colonnes ajustées
    \tkzTabInit[lgt=2.5]{$x$ / 1, $-5x+1$ / 0.8, $x^2 - 3x + 2$ / 0.8, $x^2 + 2x + 1$ / 0.8}
      {$-\infty$, $-1$, $\frac{1}{5}$, $1$, $2$, $+\infty$}

    % Ligne -5x+1
    \tkzTabLine{ , + , t , + , z , - , t ,-, t ,- }
    % Ligne x^2-3x+8
    \tkzTabLine{ , + , t , + , t , + ,z,-,z,+ }
    % Ligne x^2+2x+1
    \tkzTabLine{ , + , z , + , t , + , t ,+, t ,+ }
\end{tikzpicture}
\end{flushleft}

\[
\textcolor{green}{\boxed{S = \emptyset  }} 
\]

\item \( \sqrt{x^{2}-x-2} > x-1 \)

\[
\sqrt{x^2 - x - 2} > x - 1 \implies \textbf{(S1):}
\begin{cases}
x^2 - x - 2 \geq 0 \quad \encercle{1} \\
x - 1 \geq 0 \quad \encercle{2} \\
x^2 - x - 2 > (x-1)^2 \quad \encercle{3}
\end{cases}\textbf{ ou }
\textbf{(S2):}\begin{cases}
x - 1 < 0 \quad \encercle{4} \\
x^2 - x - 2 > 0 \quad \encercle{5}
\end{cases}
\]
\underline{\textbf{Pour (S1):}}

\(
\encercle{3}\quad x^2 - x - 2 \geq (x-1)^2 , \\
\quad x^2 - x - 2 \geq x^{2}-2x+1\\
\quad x - 3 \geq 0
\)


\[\textbf{(S1):} \quad \text{devient} \quad \textbf{(S1):}
\begin{cases}
x^2 - x - 2 \geq 0  \quad \encercle{1}\\
x - 1 \geq 0  \quad \encercle{2}\\
\quad x - 3 > 0 \quad \encercle{3}
\end{cases}
\]

\[\text{Posons }\textbf{(S1):}
\begin{cases}
x^2 - x - 2 = 0  \quad \encercle{1}\\
x - 1  = 0 \quad \encercle{2}\\
\quad x - 3 = 0 \quad \encercle{3}
\end{cases}
\]

\(\encercle{1} \quad x^2 - x - 2 = 0 \implies x=-1 \textbf{ ou } x=2\)

\(\encercle{2} \quad x - 1 = 0 \implies x=1 \)

\(\encercle{3} \quad x - 3 = 0 \implies x=3 \)
\begin{flushleft}
\begin{tikzpicture}
    % Initialisation du tableau avec colonnes ajustées
    \tkzTabInit[lgt=2.5]{$x$ / 1, $x^2 - x - 2$ / 0.8, $x - 1$ / 0.8, $x - 3$ / 0.8}
      {$-\infty$, $-1$, $1$, $2$, $3$, $+\infty$}

    % Ligne -5x+1
    \tkzTabLine{ , + , z , - , t , - ,z,+,t,+ }
    % Ligne x^2-3x+8
    \tkzTabLine{ , - , t , - , z , + , t ,+, t,+ }
    % Ligne x^2+2x+1
    \tkzTabLine{ , - , t , - , t , - , t ,-,z ,+ }
\end{tikzpicture}
\end{flushleft}
\[
\textcolor{green}{\boxed{S1 = \left]3 ; +\infty\right[  }} 
\]
\underline{\textbf{Pour (S2):}}
\[
\textbf{(S2):}\begin{cases}
x - 1 < 0 \quad \encercle{4} \\
x^2 - x - 2 > 0 \quad \encercle{5}
\end{cases}
\]
\[\text{Posons }\textbf{(S2):}
\begin{cases}
x - 1 = 0 \quad \encercle{4} \\
x^2 - x - 2 = 0 \quad \encercle{5}
\end{cases}
\]

\( \encercle{4} x - 1 = 0 \implies x=1\)

\( \encercle{5} x^2 - x - 2 = 0 \implies x=-1 \textbf{ ou } x= 2\)
\begin{flushleft}
\begin{tikzpicture}
    % Initialisation du tableau avec colonnes ajustées
    \tkzTabInit[lgt=2.5]{$x$ / 1, $x - 1$ / 0.8, $x^2 - x - 2$ / 0.8}
      {$-\infty$, $-1$, $1$, $2$, $+\infty$}

    % Ligne -5x+1
    \tkzTabLine{ , - , t , - , z , + ,t ,+, }
    % Ligne x^2-x-2
    \tkzTabLine{ , + , z , - , t , - ,z ,+ }
\end{tikzpicture}
\end{flushleft}
\[
\textcolor{green}{\boxed{S2 = \left]-\infty ; -1\right[  }} 
\]
\[
\textcolor{green}{\boxed{S=S1 \cup S2 = \left]3 ; +\infty\right[ \cup \left]-\infty ; -1\right[  }} 
\]
\item $x^4 + x^2 - 12 \leq 0$

Posons $x^4 + x^2 - 12 = 0$ et  $X=x^2$

donc $X^4 + X - 12 = 0$

$\Delta = 49$

$X_{1}=\frac{-1+7}{2}$,$X_{1}=\frac{-1-7}{2}$

$X_{1}=3$,$X_{1}=-4 	$
\[
\textcolor{green}{\boxed{S = \{ \sqrt{3} ; -\sqrt{3} \}   }} 
\]
\end{enumerate}
    \item Résolvons dans $\mathbb{R}^2$ le système suivant
    \[
    \begin{cases}
        x^3 + y^3 = 9 \\
        \frac{x}{y} + \frac{y}{x} = \frac{5}{2}
    \end{cases} \iff
        \begin{cases}
        (x+y)(x^{2}-xy+y^{2}) = 9 \\
        \frac{x^{2}+y^{2}}{xy} = \frac{5}{2}
    \end{cases}\iff
            \begin{cases}
        (x+y)(x^{2}-xy+y^{2}) = 9 \\
        \frac{(x+y)^{2}-2xy}{xy} = \frac{5}{2}
    \end{cases}
    \]
    
\[
\begin{cases}
(x+y)(x^{2}-xy+y^{2}) = 9 \\
(x+y)^{2}-2xy = \frac{5}{2}xy
\end{cases}\iff
\begin{cases}
(x+y)(x^{2}-xy+y^{2}) = 9 \\
2(x+y)^{2}-4xy = 5xy
\end{cases}
\]
\[
\begin{cases}
(x+y)(x^{2}-xy+y^{2}) = 9 \\
2(x+y)^{2}-4xy = 5xy
\end{cases}\iff
\begin{cases}
(x+y)\left[ (x+y)^{2}-3xy\right]  = 9 \encercle{1}\\
2(x+y)^{2} = 9xy \encercle{2}
\end{cases}
\] 
\[
\begin{cases}
(S)\left[ (S)^{2}-3P\right]  = 9 \encercle{1}\\
2(S)^{2} = -9P \encercle{2}
\end{cases}\implies 
\begin{cases}
S^{3}-3PS\  = 9\encercle{1}\\
2S^{3} -9PS = 0 \encercle{2}
\end{cases}
\] 

\[
\underline{\begin{cases}
3S^{3}-9PS  = 27 \encercle{1}\\
2S^{3} -9PS = 0 \encercle{2}
\end{cases}}
\]

\[
S^{3}= 27\\
\]
\[
S^{3}=27\implies S=3
\]
\[
\textcolor{green}{\underline{ S=3 }}
\]
\[
\text{S=3 dans } \encercle{2}: 2(S)^{2} - 9P =0
\]

\[
 2(3)^{2} - 9P =0 \implies   18 - 9P =0 \implies P=1
\]
\[
\textcolor{green}{\underline{ P=2 }}
\]
On a : $X^{2}-SX+P=0$

$X^{2}-3X+2=0$

$\Delta = 1$

$X_{1}=1$ , $X_{1}=2$
\[
\textcolor{green}{\boxed{S = \{ (1,2);(2,1) \}   }} 
\]
    \item Résolvons dans $\mathbb{R}^3$ par la méthode du Pivot de Gauss le système
    \[
    \begin{cases}
        x - 2y + z = 6 \\
        -2x + y - z = -6 \\
        3x - y - 2z = -2
    \end{cases}
    \]

Choisison $x - 2y + z = 6$ comme pivot 
    \[
\underline{    \begin{cases}
        2x - 4y + 2z = 12 \\
        -2x + y - z = -6 \\
    \end{cases}}
    \]
    \[-3y+z=6\]
    \[
\underline{    \begin{cases}
        3x - 6y + 3z = 18 \\
        3x - y - 2z = -2 
    \end{cases}}
    \]
    \[-5y+5z=20\]
    \[-y+z=4\]

    \[
\begin{cases}
        -3y+z=6 \\
        -y+z=4
    \end{cases}
    \]
        \[
    \begin{cases}
        x - 2y + z = 6 \\
        \quad -3y+z=6 \\
        \quad -y+z=4
    \end{cases}
    \]
    \[
\underline{    \begin{cases}
        -3y+z=6 \\
        -3y+3z=12
    \end{cases}}
    \]
    \[-2z=-6\]
    \[z=3\]
    \[
    \begin{cases}
        x - 2y + z = 6 \\
        \quad -3y+z=6 \\
        \quad\quad\quad\quad z=3
    \end{cases}
    \]
    
    \[z=3 \text{ dans } -3y+z=6  \text{ on a } -3y+3=6 \implies y=-1\]
    \[\text{$z=3 $ et $y=-1$ dans } x - 2y + z = 6  \text{ on a } x - 2(-1) + 3 = 6 \implies x=1\]
\[
\textcolor{green}{\boxed{S = (1,-1,3)   }} 
\]
\end{enumerate}

\section*{\underline{Exercice 4 :} 1,5 pt [\textit{ Exercice d'application 1 du cours sur produit scalaire} ]} 
Soient $A(-2, -3)$, $B(1, 1)$, $C(-3, -1)$, $D(4, 2)$, $E(-4, -3)$ et $E(2, -1)$ dans un repère orthonormé.

En utilisant deux méthodes différentes (une méthode par triangle), montrer que le triangle $ABC$ est un triangle rectangle en $C$, mais que $FDE$ ne l'est pas en $E$.
\end{document}