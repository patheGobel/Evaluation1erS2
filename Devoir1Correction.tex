\documentclass[12pt,a4paper]{article}
\usepackage{amsmath,amssymb,mathrsfs,tikz,times,pifont}
\usepackage{enumitem}
%+++++++++++++++++++++++++++++++++++++++++++++++++
\usepackage{stmaryrd}
\usepackage{graphicx} % Pour l'insertion d'images
\usepackage{float}    % Pour contrôler précisément le placement
\usepackage[utf8]{inputenc}

\usepackage[french]{babel}
\usepackage[T1]{fontenc}
\usepackage{hyperref}
\usepackage{verbatim}

\usepackage{color, soul}

\usepackage{pgfplots}
\pgfplotsset{compat=1.15}
\usepackage{mathrsfs}

\usepackage{amsmath}
\usepackage{amsfonts}
\usepackage{amssymb}
\usepackage{tkz-tab}
%+++++++++++++++++++++++++++++++++++++++++++++++++
\newcommand\circitem[1]{%
\tikz[baseline=(char.base)]{
\node[circle,draw=gray, fill=red!55,
minimum size=1.2em,inner sep=0] (char) {#1};}}
\newcommand\boxitem[1]{%
\tikz[baseline=(char.base)]{
\node[fill=cyan,
minimum size=1.2em,inner sep=0] (char) {#1};}}
\setlist[enumerate,1]{label=\protect\circitem{\arabic*}}
\setlist[enumerate,2]{label=\protect\boxitem{\alph*}}
%%%::::::by chnini ameur :::::::%%%
\everymath{\displaystyle}
\usepackage[left=1cm,right=1cm,top=1cm,bottom=1.7cm]{geometry}
\usepackage{array,multirow}
\usepackage[most]{tcolorbox}
\usepackage{varwidth}
\tcbuselibrary{skins,hooks}
\usetikzlibrary{patterns}
%%%::::::by chnini ameur :::::::%%%
\newtcolorbox{exa}[2][]{enhanced,breakable,before skip=2mm,after skip=5mm,
colback=yellow!20!white,colframe=black!20!blue,boxrule=0.5mm,
attach boxed title to top left ={xshift=0.6cm,yshift*=1mm-\tcboxedtitleheight},
fonttitle=\bfseries,
title={#2},#1,
% varwidth boxed title*=-3cm,
boxed title style={frame code={
\path[fill=tcbcolback!30!black]
([yshift=-1mm,xshift=-1mm]frame.north west)
arc[start angle=0,end angle=180,radius=1mm]
([yshift=-1mm,xshift=1mm]frame.north east)
arc[start angle=180,end angle=0,radius=1mm];
\path[left color=tcbcolback!60!black,right color = tcbcolback!60!black,
middle color = tcbcolback!80!black]
([xshift=-2mm]frame.north west) -- ([xshift=2mm]frame.north east)
[rounded corners=1mm]-- ([xshift=1mm,yshift=-1mm]frame.north east)
-- (frame.south east) -- (frame.south west)
-- ([xshift=-1mm,yshift=-1mm]frame.north west)
[sharp corners]-- cycle;
},interior engine=empty,
},interior style={top color=yellow!5}}
%%%%%%%%%%%%%%%%%%%%%%%

\usepackage{fancyhdr}
\usepackage{eso-pic}         % Pour ajouter des éléments en arrière-plan
% Commande pour ajouter du texte en arrière-plan
\AddToShipoutPicture{
    \AtTextCenter{%
        \makebox[0pt]{\rotatebox{80}{\textcolor[gray]{0.7}{\fontsize{5cm}{5cm}\selectfont PGB}}}
    }
}
\usepackage{lastpage}
\fancyhf{}
\pagestyle{fancy}
\renewcommand{\footrulewidth}{1pt}
\renewcommand{\headrulewidth}{0pt}
\renewcommand{\footruleskip}{10pt}
\fancyfoot[R]{
\color{blue}\ding{45}\ \textbf{2024}
}
\fancyfoot[L]{
\color{blue}\ding{45}\ \textbf{Prof:M. BA}
}
\cfoot{\bf
\thepage /
\pageref{LastPage}}
\begin{document}
\renewcommand{\arraystretch}{1.5}
\renewcommand{\arrayrulewidth}{1.2pt}
\begin{tikzpicture}[overlay,remember picture]
\node[draw=blue,line width=1.2pt,fill=purple,text=blue,inner sep=3mm,rounded corners,pattern=dots]at ([yshift=-2.5cm]current page.north) {\begingroup\setlength{\fboxsep}{0pt}\colorbox{white}{\begin{tabular}{|*1{>{\centering \arraybackslash}p{0.28\textwidth}} |*2{>{\centering \arraybackslash}p{0.2\textwidth}|} *1{>{\centering \arraybackslash}p{0.19\textwidth}|} }
\hline
\multicolumn{3}{|c|}{$\diamond$$\diamond$$\diamond$\ \textbf{Lycée de Dindéfélo}\ $\diamond$$\diamond$$\diamond$ }& \textbf{A.S. : 2024/2025} \\ \hline
\textbf{Matière: Mathématiques}& \textbf{Niveau : 1}\textbf{$^{er}$S2} &\textbf{Date: 14/12/2024} & \textbf{Durée : 4 heures} \\ \hline
\multicolumn{4}{|c|}{\parbox[c]{10cm}{\begin{center}
\textbf{{\Large\sffamily Devoir n$ ^{\circ} $ 1 Du 1$ ^\text{\bf er} $ Semestre}}
\end{center}}} \\ \hline
\end{tabular}}\endgroup};
\end{tikzpicture}
\vspace{3cm}

\section*{\underline{Exercice 1 :} 3 points}
Soit \( m \) un nombre réel. On considère l’équation d’inconnue \( x \) :

\( (E) \) \( (m+3)x^{2} +2mx + m + 5 = 0\)
\begin{enumerate}

\item[a)] Discuter suivant les valeurs de \( m \), l’existence, le nombre et le signe des racines de \( (E) \) 

Dans le cas où \( (E) \) admet deux racines distinctes \( x_{2} \) et \( x_{2} \) , peut-on déterminer \( m \) pour que:

\( ( 2x_{1} - 1 ) (2x_{2}-1) = 6 \) puis \( x_{1}^{2} + x_{2}^{2} = 2 \)

\item[b)]Trouver une relation indépendante de \( m \) entre \( x_{2} \) et \( x_{2} \)
\end{enumerate}
\section*{\underline{Correction Exercice 1 :} 3 points}
Soit \( m \) un nombre réel. On considère l’équation d’inconnue \( x \) :

\( (E_{m}) \):\( (m+3)x^{2} +2mx + m + 5 = 0\)
\begin{enumerate}

\item[a)] Discutons suivant les valeurs de \( m \), l’existence, le nombre et le signe des racines de \( (E) \) 

\begin{itemize}
\item si \( (m+3) = 0 \implies m = -3 \) alors l'équation \( (E_{m}) \) est du $1^{\text{er}}$ degré et devient\( (E_{-3}) \)\\
\( (E_{-3}) \) : \(  2(-3)x - 3 + 5 = 0\)\\
\( (E_{-3}) \) : \(  -6x +2 = 0\)\\
\( (E_{-3}) \) : \(  x = \frac{1}{3}\) \textbf{0,5 pt}
\item si \( (m+3) \neq 0 \implies m \neq -3 \) alors l'équation \( (E_{m}) \) est du $2^{\text{nd}}$ degré\\
Chechons \(\Delta_{m}'\)

\begin{align*}
\Delta_{m}'&=(m)^{2}-(m+3)(m+5)\\
					&=m^{2}-(m^{2}+5m+3m+15)\\
					&=m^{2}-m^{2}-8m-15\\
					&=-8m-15\\
\end{align*}
Chechons le signe de \(\Delta_{m}'\). Pour ce faire, posons \(\Delta_{m}'=0\)

\( \Delta_{m}'=0 \implies -8m-15=0 \implies m=\frac{-15}{8}\) 
\end{itemize}
\vspace{0.5cm}

\noindent \textbf{Tableau de signe :}

\begin{tikzpicture}
    \tkzTabInit[espcl=2, lgt=5] % `lgt` augmente globalement la taille des colonnes
    {$m$ / 1 , $-8m-15$ / 1}
    {$-\infty$, $-\frac{15}{8}$, $+\infty$}
    \tkzTabLine{, +, z, -, }
\end{tikzpicture}

\vspace{0.5cm}

\noindent \textbf{Résultats :}

\begin{itemize}
    \item Si $m \in \left]-\frac{15}{8}, +\infty\right[ \Rightarrow \Delta_m' < 0$ donc $(E_m)$ n'admet pas de solution et $S = \emptyset$.
    \item Si $m = -\frac{15}{8} \Rightarrow \Delta_m' = 0$ donc $(E_m)$ admet une racine double $x_{0}=\frac{-m}{m+3}$.

		\item \( m \in \left]-\infty, -\frac{15}{8} \right[ \setminus \{-3\} \) alors \( \Delta_m > 0 \), donc :

L'équation (Em) admet deux racines \(x_1\) et \(x_2\) telles que :
\[
x_1 = \frac{-m - \sqrt{\Delta_m'}}{m+3}, \quad x_2 = \frac{-m + \sqrt{\Delta_m'}}{m+3}
\]
\end{itemize}

\textbf{Dans le cas où \( \Delta_m > 0 \)}, déterminons \( m \) pour que :
\[
(2x_1 - 1)(2x_2 - 1) = 6 \quad \text{puisque} \quad x_1 + x_2 = 2
\]
\begin{itemize}

\item $ \underline{(2x_1 - 1)(2x_2 - 1) = 6} $

\[
2x_1 (2x_2 - 1) - (2x_2 - 1) = 6
\]

\[
4x_1 x_2 - 2x_1 - 2x_2 + 1 = 6
\]

\[
4x_1 x_2 - 2(x_1 + x_2) = 5
\]

\[
4P - SP = 5
\]

\[ \text{avec }
S = \frac{-2m}{m+3} \quad \text{et} \quad P = \frac{m+5}{m+3}
\]

\[
4 \left( \frac{m+5}{m+3} \right) - 2 \left( \frac{-2m}{m+3} \right) = 5
\]

\[
\frac{4m + 20 + 4m}{m+3} = 5
\]

\[
4m + 20 + 4m = 5m + 15
\]

\[
3m = -5
\]

\[
m = -\frac{5}{3}
\]

\textbf{Conclusion} :  
Pour que \( (2x_1 - 1)(2x_2 - 1) = 6 \) soit vérifiée, il faut que :
\(
m = -\frac{5}{3}
\)

\item $ \underline{x_1^{2} + x_2^{2} = 2} $

\end{itemize}
\end{enumerate}

\end{document}