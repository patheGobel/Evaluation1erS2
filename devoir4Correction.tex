\documentclass[12pt,a4paper]{article}
\usepackage{amsmath,amssymb,mathrsfs,tikz,times,pifont}
\usepackage{enumitem}
\usepackage{float}
\newcommand\circitem[1]{%
\tikz[baseline=(char.base)]{
\node[circle,draw=gray, fill=red!55,
minimum size=1.2em,inner sep=0] (char) {#1};}}
\newcommand\boxitem[1]{%
\tikz[baseline=(char.base)]{
\node[fill=cyan,
minimum size=1.2em,inner sep=0] (char) {#1};}}
\setlist[enumerate,1]{label=\protect\circitem{\arabic*}}
\setlist[enumerate,2]{label=\protect\boxitem{\alph*}}
%%%::::::by chnini ameur :::::::%%%
\everymath{\displaystyle}
\usepackage[left=1cm,right=1cm,top=1cm,bottom=1.7cm]{geometry}
\usepackage{array,multirow}
\usepackage[most]{tcolorbox}
\usepackage{varwidth}
\tcbuselibrary{skins,hooks}
\usetikzlibrary{patterns}
%%%::::::by chnini ameur :::::::%%%
\newtcolorbox{exa}[2][]{enhanced,breakable,before skip=2mm,after skip=5mm,
colback=yellow!20!white,colframe=black!20!blue,boxrule=0.5mm,
attach boxed title to top left ={xshift=0.6cm,yshift*=1mm-\tcboxedtitleheight},
fonttitle=\bfseries,
title={#2},#1,
% varwidth boxed title*=-3cm,
boxed title style={frame code={
\path[fill=tcbcolback!30!black]
([yshift=-1mm,xshift=-1mm]frame.north west)
arc[start angle=0,end angle=180,radius=1mm]
([yshift=-1mm,xshift=1mm]frame.north east)
arc[start angle=180,end angle=0,radius=1mm];
\path[left color=tcbcolback!60!black,right color = tcbcolback!60!black,
middle color = tcbcolback!80!black]
([xshift=-2mm]frame.north west) -- ([xshift=2mm]frame.north east)
[rounded corners=1mm]-- ([xshift=1mm,yshift=-1mm]frame.north east)
-- (frame.south east) -- (frame.south west)
-- ([xshift=-1mm,yshift=-1mm]frame.north west)
[sharp corners]-- cycle;
},interior engine=empty,
},interior style={top color=yellow!5}}
%%%%%%%%%%%%%%%%%%%%%%%

\newtcolorbox{resultbox}{
    colback=red!30, % Fond rouge clair
    colframe=black, % Bordure noire fine
    sharp corners, % Coins nets
    boxrule=0.5pt, % Contour l\'eger
    boxsep=2pt, % Espacement interne
    left=5pt, right=5pt, top=2pt, bottom=2pt, % Marges internes
}

\usepackage{fancyhdr}
\usepackage{eso-pic}         % Pour ajouter des éléments en arrière-plan
% Commande pour ajouter du texte en arrière-plan
\AddToShipoutPicture{
    \AtTextCenter{%
        \makebox[0pt]{\rotatebox{80}{\textcolor[gray]{0.7}{\fontsize{5cm}{5cm}\selectfont PGB}}}
    }
}
\usepackage{lastpage}
\fancyhf{}
\pagestyle{fancy}
\renewcommand{\footrulewidth}{1pt}
\renewcommand{\headrulewidth}{0pt}
\renewcommand{\footruleskip}{10pt}
\fancyfoot[R]{
\color{blue}\ding{45}\ \textbf{2025}
}
\fancyfoot[L]{
\color{blue}\ding{45}\ \textbf{Prof:M. BA}
}
\cfoot{\bf
\thepage /
\pageref{LastPage}}
\begin{document}
\renewcommand{\arraystretch}{1.5}
\renewcommand{\arrayrulewidth}{1.2pt}
\begin{tikzpicture}[overlay,remember picture]
    \node[draw=blue,line width=1.2pt,fill=purple,text=blue,inner sep=3mm,rounded corners,pattern=dots]at ([yshift=-2.5cm]current page.north) {\begingroup\setlength{\fboxsep}{0pt}\colorbox{white}{\begin{tabular}{|*1{>{\centering \arraybackslash}p{0.28\textwidth}} |*2{>{\centering \arraybackslash}p{0.2\textwidth}|} *1{>{\centering \arraybackslash}p{0.19\textwidth}|} }
                \hline
                \multicolumn{3}{|c|}{$\diamond$$\diamond$$\diamond$\ \textbf{Lycée de Dindéfélo}\ $\diamond$$\diamond$$\diamond$ } & \textbf{A.S. : 2024/2025}                                                                     \\ \hline
                \textbf{Matière: Mathématiques}                                                                                    & \textbf{Niveau : 1}\textbf{$^{er}$S2} & \textbf{Date: 10/05/2025} & \textbf{Durée : 4 heures} \\ \hline
                \multicolumn{4}{|c|}{\parbox[c]{10cm}{\begin{center}
                                                                  \textbf{{\Large\sffamily Devoir n$ ^{\circ} $ 2 Du 2$ ^\text{\bf nd} $ Semestre}}
                                                              \end{center}}}                                                                                                                               \\ \hline
            \end{tabular}}\endgroup};
\end{tikzpicture}
\vspace{3cm}

\section*{\underline{Exercice 1 :} 8 pts }
Calculer les limites suivantes :

\(
\begin{aligned}
\textbf{a. } \lim_{x \to +\infty} \frac{-x^2 + 3}{3x^4 + x + 2} &= \lim_{x \to +\infty} \frac{-x^2}{3x^4}\\
																																&= \lim_{x \to +\infty} \frac{-1}{3x^2}\\
																																&=0
\end{aligned}
\)

									  \begin{resultbox}
                        \[
                            \mathbf{\lim_{x \to +\infty} \frac{-x^2 + 3}{3x^4 + x + 2} = 0 }
                        \]
                    \end{resultbox}	

\(
\begin{aligned}
\textbf{b. } \lim_{x \to 1} \frac{x^2 + x - 2}{x - 1} & = \lim_{x \to 1} \frac{(x-1)(x+2)}{x - 1}\\
																											& = \lim_{x \to 1} (x+2)\\
																											& = 3
\end{aligned}
\)

									  \begin{resultbox}
                        \[
                            \mathbf{ \lim_{x \to 1} \frac{x^2 + x - 2}{x - 1} = 3 }
                        \]
                    \end{resultbox}	


\(
\begin{aligned}
\textbf{c. } \lim_{x \to -\infty} \sqrt{x^{2}+3}+x & =  \lim_{x \to -\infty} \frac{3}{\sqrt{x^{2}+3}-x}\\
& =  0
\end{aligned}
\)

									  \begin{resultbox}
                        \[
                            \mathbf{ \lim_{x \to -\infty} \sqrt{x^{2}+3}+x = 0 }
                        \]
                    \end{resultbox}	
                    
                    
\(
\begin{aligned}
\textbf{d. }\lim_{x \to +\infty} \frac{\sqrt{x^2 - 4}}{x - 2} & =  \lim_{x \to +\infty} \frac{\sqrt{x^2\left(1 - \frac{4}{x^{2}}\right)}}{x - 2}\\
& =  \lim_{x \to +\infty} \frac{|x|\sqrt{\left(1 - \frac{4}{x^{2}}\right)}}{x\left(1 - \frac{2}{x}\right)}\\
& =  \lim_{x \to +\infty} \frac{\sqrt{\left(1 - \frac{4}{x^{2}}\right)}}{\left(1 - \frac{2}{x}\right)}\\
& = 1
\end{aligned}
\)

									  \begin{resultbox}
                        \[
                            \mathbf{ \lim_{x \to +\infty} \frac{\sqrt{x^2 - 4}}{x - 2} = 1 }
                        \]
                    \end{resultbox}	
                    
\(
\begin{aligned}
\textbf{e. }\lim_{x \to +\infty} \frac{x}{\sqrt{x^2 +1}-1} &= \lim_{x \to +\infty} \frac{x(\sqrt{x^2 +1}+1)}{x^2 +1-1}\\
																													 &= \lim_{x \to +\infty} \frac{x(\sqrt{x^2 +1}+1)}{x^2}\\
																													 &= \lim_{x \to +\infty} \frac{\sqrt{x^2 +1}+1}{x}\\
																													 &= \lim_{x \to +\infty} \frac{x\left(\sqrt{1 + \frac{1}{x^{2}}}+\frac{1}{x}\right)}{x}\\																													 
																													 &= \lim_{x \to +\infty}\left(\sqrt{1 + \frac{1}{x^{2}}}+\frac{1}{x}\right)\\
																													 &= 1
\end{aligned}
\)

									  \begin{resultbox}
                        \[
                            \mathbf{ \lim_{x \to +\infty} \frac{x}{\sqrt{x^2 +1}-1} = 1 }
                        \]
                    \end{resultbox}	
                    
\(
\begin{aligned}
\textbf{f. }\lim_{x \to 4} \frac{\sqrt{2x + 1} - 3}{x - 4} &= \lim_{x \to 4} \frac{\left(\sqrt{2x + 1} - 3\right)\left(\sqrt{2x + 1} + 3\right)}{(x - 4)\left(\sqrt{2x + 1} + 3\right)}\\
																													 &= \lim_{x \to 4} \frac{\left(2x + 1 - 9\right)}{(x - 4)\left(\sqrt{2x + 1} + 3\right)}\\
																													 &= \lim_{x \to 4} \frac{\left(2x - 8\right)}{(x - 4)\left(\sqrt{2x + 1} + 3\right)}\\
																													 &= \lim_{x \to 4} \frac{2\left(x - 4\right)}{(x - 4)\left(\sqrt{2x + 1} + 3\right)}\\
																													 &= \lim_{x \to 4} \frac{2}{\left(\sqrt{2x + 1} + 3\right)}\\
																													 &=\frac{1}{3}
\end{aligned}
\)

									  \begin{resultbox}
                        \[
                            \mathbf{ \lim_{x \to 4} \frac{\sqrt{2x + 1} - 3}{x - 4} = \frac{1}{3}}
                        \]
                    \end{resultbox}	
                    
\(
\begin{aligned}
\textbf{g. }\lim_{x \to 0} \frac{\sqrt{4 - x} - \sqrt{4 + x}}{x} &= \lim_{x \to 0} \frac{4 - x - 4 - x}{x\left(\sqrt{4 - x} + \sqrt{4 + x}\right)}\\
																																 &= \lim_{x \to 0} \frac{-2x}{x\left(\sqrt{4 - x} + \sqrt{4 + x}\right)}\\
																																 &= \lim_{x \to 0} \frac{-2}{\left(\sqrt{4 - x} + \sqrt{4 + x}\right)}\\
																																 &= \dfrac{-1}{2}
\end{aligned}
\)

									  \begin{resultbox}
                        \[
                            \mathbf{ \lim_{x \to 0} \frac{\sqrt{4 - x} - \sqrt{4 + x}}{x} = \frac{-1}{2} }
                        \]
                    \end{resultbox}	
                    
\(
\begin{aligned}
\textbf{h. }\lim\limits_{x \to -\infty} \frac{\sqrt{3x^2 + 3x + 2}}{2x + 3} &= \lim\limits_{x \to -\infty} \frac{\sqrt{x^{2}\left(3 + \frac{3}{x} + \frac{2}{x^{2}}\right)}}{x\left(2 + \frac{3}{x} \right)}\\
																																						&= \lim\limits_{x \to -\infty} \frac{|x|\sqrt{\left(3 + \frac{3}{x} + \frac{2}{x^{2}}\right)}}{x\left(2 + \frac{3}{x} \right)}\\
																																						&= \lim\limits_{x \to -\infty} \frac{-x\sqrt{\left(3 + \frac{3}{x} + \frac{2}{x^{2}}\right)}}{x\left(2 + \frac{3}{x} \right)}\\
																																						&= \lim\limits_{x \to -\infty} \frac{-\sqrt{\left(3 + \frac{3}{x} + \frac{2}{x^{2}}\right)}}{\left(2 + \frac{3}{x} \right)}\\
																																						&=\frac{-\sqrt{3}}{2}
\end{aligned}
\)

									  \begin{resultbox}
                        \[
                            \mathbf{ \lim\limits_{x \to -\infty} \frac{\sqrt{3x^2 + 3x + 2}}{2x + 3} = \frac{-\sqrt{3}}{2} }
                        \]
                    \end{resultbox}	
                   
\section*{\underline{Exercice 2 :} 8 pts }
Soit \( g \) la fonction définie par :
\(
g(x) =
\begin{cases}
    x + \sqrt{|x^2 - x|},     & \text{si } x \geq 0 \\
    \frac{x - x(x-2)}{x - 1}, & \text{si } x < 0
\end{cases}
\)

\begin{enumerate}
    \item Justifions que la fonction \( g \) est définie sur \( D_g = \mathbb{R} \).
    
                     \( \text{Posons }g(x) = \left\{
                    \begin{array}{ll}
                        g_{1}(x) & \text{si } x \geq 0    \\
                        g_{2}(x)         & \text{si } x < 0
                    \end{array}
                    \right. \)
                    
                    \(
                    \begin{aligned}
                    	 g_{1} \exists \text{ ssi }&  |x^2 - x| \geq 0  &\text{ et }&  x \geq 0 \\
                    								  \text{ ssi }&			x\in \mathbb{R}  &\text{ et }&  x\in[0,+\infty[ \\
                    								  \text{ ssi }&			 x\in[0,+\infty[\\
                    								  						&      \underline{Dg_{1} = [0,+\infty[}
                    \end{aligned}
										\)
										
										\(
									\begin{aligned}
											g_{2} \exists \text{ ssi }&  x - 1 \neq 0  &\text{ et }&  x < 0 \\
																		\text{ ssi }& x  \neq 1  &\text{ et }&  x\in]-\infty,0[ \\
																		\text{ ssi }&	x\in]-\infty,0[\\
																		&\underline{ Dg_{2} = ]-\infty,0[ }
									\end{aligned}
									\)

\[ Dg= \mathbb{R} \]									
									
									  \begin{resultbox}
                        \[
                            \mathbf{Dg= \mathbb{R} }
                        \]
                    \end{resultbox}	
    \item Déterminons les limites aux bornes de l'ensemble de définition.

		Les bornes de $Dg$ sont $-\infty,+\infty$ 
		
		\underline{En $+\infty$ :} $g(x) = g_{1}(x)$

										\(
									\begin{aligned}
											\lim\limits_{x \to +\infty}g(x) &= \lim\limits_{x \to +\infty}g_{1}(x)\\
																											&= \lim\limits_{x \to +\infty}x + \sqrt{|x^2 - x|}\\
																											&=+\infty
									\end{aligned}
									\)    
 
										\begin{resultbox}
                        \[
                            \mathbf{ \underline{\lim\limits_{x \to +\infty}g(x) = +\infty} }
                        \]
                    \end{resultbox}   
    
    		\underline{En $-\infty$ :} $g(x) = g_{2}(x)$

										\(
									\begin{aligned}
											\lim\limits_{x \to -\infty}g(x) &= \lim\limits_{x \to -\infty}g_{2}(x)\\
																											&= \lim\limits_{x \to -\infty}\frac{x - x(x-2)}{x - 1}\\
																											&=\lim\limits_{x \to -\infty}-\frac{x^{2}}{x}\\
																											&=\lim\limits_{x \to -\infty}-x\\
																											&=+\infty
									\end{aligned}
									\)    

											\begin{resultbox}
                        \[
                            \mathbf{ \underline{\lim\limits_{x \to -\infty}g(x) = +\infty} }
                        \]
                    \end{resultbox}
    \item Étudions la continuité de \( g \) sur \( [0, +\infty[ \)    
    
\( g_{1}(x) = x + \sqrt{|x^2 - x|} \) est une fonction irrationnelle donc continue sur son domaine $Dg_{1} = [0,+\infty[$
                    
    \item Étudions la continuité de \( g \) sur \( ]-\infty, 0[ \)

\( g_{2}(x) = \frac{x - x(x-2)}{x - 1} \) est une fonction rationnelle donc continue sur son domaine $Dg_{2} = ]-\infty, 0[$

    \item Étudions  la continuité de \( g \) en \( 0 \)

\textbf{A-t-on $ \lim\limits_{x \to 0^{+}}g(x) = \lim\limits_{x \to 0^{-}}g(x) = g(0) $ ?}

\textbf{$g(0)=0$ }   
    
    \underline{En $0^{+}$ :} $g(x) = g_{1}(x)$

										\(
									\begin{aligned}
											\lim\limits_{x \to 0^{+}}g(x) &= \lim\limits_{x \to 0^{+}}g_{1}(x)\\
																											&= \lim\limits_{x \to 0^{+}}x + \sqrt{|x^2 - x|}\\
																											&=0
									\end{aligned}
									\)    
 
										\begin{resultbox}
                        \[
                            \mathbf{ \underline{\lim\limits_{x \to 0^{+}}g(x) = 0} }
                        \]
                    \end{resultbox}       

\underline{En $0^{-}$ :} $g(x) = g_{2}(x)$

										\(
									\begin{aligned}
											\lim\limits_{x \to 0^{-}}g(x) &= \lim\limits_{x \to 0^{-}}g_{2}(x)\\
																											&= \lim\limits_{x \to 0^{-}}\frac{x - x(x-2)}{x - 1}\\
																											&=0
									\end{aligned}
									\)    

											\begin{resultbox}
                        \[
                            \mathbf{ \underline{\lim\limits_{x \to 0^{-}}g(x) = 0} }
                        \]
                    \end{resultbox}     
    \textbf{Conclusion : Comme $ \lim\limits_{x \to 0^{+}}g(x) = \lim\limits_{x \to 0^{-}}g(x) = g(0) $ donc la fonction est continu en $0$}  
    \item En déduire l'ensemble de continuité de \( g \)

\( g \) est continue sur  $Dg_{1} = [0,+\infty[$,  $Dg_{2} = ]-\infty, 0[$  et en $0$ donc $g$ est continu sur $\mathbb{R}$ 
    
    \item Soit \( f \) une fonction dont sa représentation graphique est ci-dessous. \( f \) est-elle continue sur \( [-3, -1] \) ? Justifier.
          % Inclusion de l'image ici
          \begin{center}
              \begin{figure}[H]
                 % \centering \includegraphics[width=0.5\textwidth]{Screenshot from 2025-04-28 19-44-54.png} % Remplacer "votre_image.png" par le nom de votre fichier image
                  \caption{Représentation graphique de la fonction \( f \).}
                  \label{fig:fonction_f}
              \end{figure}
          \end{center}
\end{enumerate}
\section*{\underline{Exercice 3 :} 4 pts }
\begin{enumerate}
    \item Déterminons le domaine de définition des fonctions suivantes :
          \begin{enumerate}
              \item  \( f(x) = \frac{(x - 3)(x + 8)}{x^2 - 2} \)

                    \(f\,\exists \) ssi \(x^2 - 2 \neq 0\)

                    \(x^2 - 2 \neq 0 \implies x \neq \sqrt{2} \text{ et } x \neq -\sqrt{2}\)

                    \( Df= \mathbb{R}\setminus\{-\sqrt{2},\sqrt{2}\} \)

                    \begin{resultbox}
                        \[
                            \mathbf{Df= \mathbb{R}\setminus\{-\sqrt{2},\sqrt{2}\} }
                        \]
                    \end{resultbox}
              \item \( g(x) = \frac{2x^2 - 8x + 2}{\sqrt{x^2 - 4}} \)

                    \(g\,\exists \) ssi \( x^2 - 4  > 0 \)

                    \( x^2 - 4  > 0 \implies x\in ]-\infty,-\sqrt{2}[ \cup ]\sqrt{2},+\infty[ \)

                        \( Df= ]-\infty,-\sqrt{2}[ \cup ]\sqrt{2},+\infty[ \)

                    \begin{resultbox}
                        \[
                            \mathbf{Df= ]-\infty,-\sqrt{2}[ \cup ]\sqrt{2},+\infty[ }
                        \]
                    \end{resultbox}
              \item \( h(x) = \left\{
                    \begin{array}{ll}
                        \frac{x^2 - 2x}{x - 1} & \text{si } x < 0    \\
                        \sqrt{x^2 + x}         & \text{si } x \geq 0
                    \end{array}
                    \right. \)

                    \( \text{Posons }h(x) = \left\{
                    \begin{array}{ll}
                        h_{1}(x) & \text{si } x < 0    \\
                        h_{2}(x)         & \text{si } x \geq 0
                    \end{array}
                    \right. \)
                    
										\(
                    \begin{aligned}
                    	 h_{1} \exists \text{ ssi }&  x - 1 \neq 0  \text{ et }  x < 0 \\
                    								  \text{ ssi }&			 x \neq 1  \text{ et }  x\in]-\infty,0[ \\
                    								  \text{ ssi }&			 x\in]-\infty,0[\\
                    								  						&      Dh_{1} = ]-\infty,0[ 
                    \end{aligned}
										\)
										
\(
\begin{aligned}
h_{2} \exists \text{ ssi }&  x^2 + x \geq 0  \text{ et }  x \geq 0 \\
\text{ ssi }& x \in ]-\infty,-1] \cup [0,+\infty[  \text{ et }  x\in[0,+\infty[ \\
\text{ ssi }&	x\in[0,+\infty[\\
& Dh_{2} = [0,+\infty[ 
\end{aligned}
\)			
						
\(
						\begin{aligned} 
						Dh&=Dh_{1} \cup Dh_{2}\\
						  &= ]-\infty,0[ \cup [0,+\infty[\\
						  &= ]-\infty,+\infty[ \\
						  &=\mathbb{R} 
						 \end{aligned}
\)	

                    \begin{resultbox}
                        \[
                            \mathbf{Dh=\mathbb{R} }
                        \]
                    \end{resultbox}										
\end{enumerate}
    \item Soit \( f(x) = \sqrt{x + 1} \) et \( g(x) = \frac{x + 2}{x - 3} \) deux fonctions.
          \begin{enumerate}
              \item Déterminer \( Df \) et \( Dg \)
                   \begin{resultbox}
                        \[
                            \mathbf{Df= [-1,+\infty[ }
                        \]
                    \end{resultbox}
                     \begin{resultbox}
                        \[
                            \mathbf{Dg=\mathbb{R}\setminus\{ 3 \} }
                        \]
                    \end{resultbox}
              \item Déterminons \( Dg \circ f \)
              
              \( 
              \begin{aligned}
              Dg \circ f &= \{ x\in Df | f(x) \in Dg \}\\
              					 &= \{ x\in [-1,+\infty[ \textbf{ et } f(x) \in \mathbb{R}\setminus\{ 3 \} \}\\
              					 &= \{ x\in [-1,+\infty[ \textbf{ et } f(x) \neq 3 \}\\ 
              					 &= \{ x\in [-1,+\infty[ \textbf{ et } \sqrt{x + 1} \neq 3 \}\\
              					 &= \{ x\in [-1,+\infty[ \textbf{ et } x+1 \neq 9 \}\\
              					 &= \{ x\in [-1,+\infty[ \textbf{ et } x \neq 8 \}\\
              					 &= \{ x\in [-1,8[ \cup ]8,+\infty[ \}\\
              					 &=[-1,8[ \cup ]8,+\infty[
              \end{aligned}
              \)
              
                    \begin{resultbox}
                        \[
                            \mathbf{Dg \circ f=[-1,8[ \cup ]8,+\infty[ }
                        \]
                    \end{resultbox}

						Calculons \( g \circ f(x) \) 
						
						\( 
              \begin{aligned} 
              g \circ f(x) &= \frac{f(x) + 2}{f(x) - 3}\\ 
              						 &= \frac{\sqrt{x + 1} + 2}{\sqrt{x + 1} - 3}\\ 
               \end{aligned}
              \)                 

                   \begin{resultbox}
                        \[
                            \mathbf{ g \circ f(x) = \frac{\sqrt{x + 1} + 2}{\sqrt{x + 1} - 3} }
                        \]
                    \end{resultbox}                    
                    
          \end{enumerate}

\end{enumerate}
\end{document}