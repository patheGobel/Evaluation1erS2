\documentclass[12pt,a4paper]{article}
\usepackage{amsmath,amssymb,mathrsfs,tikz,times,pifont}
\usepackage{enumitem}
\newcommand\circitem[1]{%
\tikz[baseline=(char.base)]{
\node[circle,draw=gray, fill=red!55,
minimum size=1.2em,inner sep=0] (char) {#1};}}
\newcommand\boxitem[1]{%
\tikz[baseline=(char.base)]{
\node[fill=cyan,
minimum size=1.2em,inner sep=0] (char) {#1};}}
\setlist[enumerate,1]{label=\protect\circitem{\arabic*}}
\setlist[enumerate,2]{label=\protect\boxitem{\alph*}}
%%%::::::by chnini ameur :::::::%%%
\everymath{\displaystyle}
\usepackage[left=1cm,right=1cm,top=1cm,bottom=1.7cm]{geometry}
\usepackage{array,multirow}
\usepackage[most]{tcolorbox}
\usepackage{varwidth}
\tcbuselibrary{skins,hooks}
\usetikzlibrary{patterns}
%%%::::::by chnini ameur :::::::%%%
\newtcolorbox{exa}[2][]{enhanced,breakable,before skip=2mm,after skip=5mm,
colback=yellow!20!white,colframe=black!20!blue,boxrule=0.5mm,
attach boxed title to top left ={xshift=0.6cm,yshift*=1mm-\tcboxedtitleheight},
fonttitle=\bfseries,
title={#2},#1,
% varwidth boxed title*=-3cm,
boxed title style={frame code={
\path[fill=tcbcolback!30!black]
([yshift=-1mm,xshift=-1mm]frame.north west)
arc[start angle=0,end angle=180,radius=1mm]
([yshift=-1mm,xshift=1mm]frame.north east)
arc[start angle=180,end angle=0,radius=1mm];
\path[left color=tcbcolback!60!black,right color = tcbcolback!60!black,
middle color = tcbcolback!80!black]
([xshift=-2mm]frame.north west) -- ([xshift=2mm]frame.north east)
[rounded corners=1mm]-- ([xshift=1mm,yshift=-1mm]frame.north east)
-- (frame.south east) -- (frame.south west)
-- ([xshift=-1mm,yshift=-1mm]frame.north west)
[sharp corners]-- cycle;
},interior engine=empty,
},interior style={top color=yellow!5}}
%%%%%%%%%%%%%%%%%%%%%%%

\usepackage{fancyhdr}
\usepackage{eso-pic}         % Pour ajouter des éléments en arrière-plan
% Commande pour ajouter du texte en arrière-plan
\AddToShipoutPicture{
    \AtTextCenter{%
        \makebox[0pt]{\rotatebox{80}{\textcolor[gray]{0.7}{\fontsize{5cm}{5cm}\selectfont PGB}}}
    }
}
\usepackage{lastpage}
\fancyhf{}
\pagestyle{fancy}
\renewcommand{\footrulewidth}{1pt}
\renewcommand{\headrulewidth}{0pt}
\renewcommand{\footruleskip}{10pt}
\fancyfoot[R]{
\color{blue}\ding{45}\ \textbf{2024}
}
\fancyfoot[L]{
\color{blue}\ding{45}\ \textbf{Prof:M. BA}
}
\cfoot{\bf
\thepage /
\pageref{LastPage}}
\begin{document}
\renewcommand{\arraystretch}{1.5}
\renewcommand{\arrayrulewidth}{1.2pt}
\begin{tikzpicture}[overlay,remember picture]
\node[draw=blue,line width=1.2pt,fill=purple,text=blue,inner sep=3mm,rounded corners,pattern=dots]at ([yshift=-2.5cm]current page.north) {\begingroup\setlength{\fboxsep}{0pt}\colorbox{white}{\begin{tabular}{|*1{>{\centering \arraybackslash}p{0.28\textwidth}} |*2{>{\centering \arraybackslash}p{0.2\textwidth}|} *1{>{\centering \arraybackslash}p{0.19\textwidth}|} }
\hline
\multicolumn{3}{|c|}{$\diamond$$\diamond$$\diamond$\ \textbf{Lycée de Dindéfélo}\ $\diamond$$\diamond$$\diamond$ }& \textbf{A.S. : 2024/2025} \\ \hline
\textbf{Matière: Mathématiques}& \textbf{Niveau : 1}\textbf{$^{er}$S2} &\textbf{Date: 14/01/2025} & \textbf{Durée : 4 heures} \\ \hline
\multicolumn{4}{|c|}{\parbox[c]{10cm}{\begin{center}
\textbf{{\Large\sffamily Devoir n$ ^{\circ} $ 2 Du 1$ ^\text{\bf er} $ Semestre}}
\end{center}}} \\ \hline
\end{tabular}}\endgroup};
\end{tikzpicture}
\vspace{3cm}

\section*{\underline{Exercice 1 :} 3 pts [\textit{ Déjà corrigé en classe par moi-même} ]}
Soit \( ABC \) un triangle tel que \( AB = 4\,\text{cm} \) et \( BC = 5\,\text{cm} \).

On donne \( I = \text{bar} \{(A,3), (B,-2)\} \) et \( J = \text{bar} \{(B,-2), (C,3)\} \), soit \( G \) le point défini par :
\[
3\overrightarrow{GA} - 2\overrightarrow{GB} + 3\overrightarrow{GC} = \overrightarrow{0}.
\]

\begin{enumerate}
    \item Montrer que \( G = \text{bar} \{(I,-1), (C,3)\} \). \hfill \textbf{0,75 pt}
    \item Déterminer et construire l'ensemble des points \( P \) du plan tels que :
    \begin{enumerate}
        \item \( \|3\overrightarrow{MA} - 2\overrightarrow{MB} + 3\overrightarrow{MC}\| = AB \),\hfill \textbf{0,75 pt}
        \item \( \|18\overrightarrow{MA} - 12\overrightarrow{MB}\| = \|\overrightarrow{MA} - \overrightarrow{MB}\| \),\hfill \textbf{0,75 pt}
        \item \( \|3\overrightarrow{MA} - 9\overrightarrow{MB}\| = \| -2\overrightarrow{MB} + 3\overrightarrow{MC}\| \).\hfill \textbf{0,75 pt}
    \end{enumerate}
\end{enumerate}

\section*{\underline{Exercice 2 :} 7 points [\textit{ Exercice 2 du devoir 1 déjà corrigé en classe } ]}

On donne \( P(x) = 2x^3 + ax^2 + bx - 6 \) où \( a \) et \( b \) sont des réels.
\begin{enumerate}
    \item Trouver les réels \( a \) et \( b \) pour que le polynôme \( P(x) \) soit divisible par \( x^2 - x - 2 \).
    \item En déduire une factorisation de \( P(x) \).
    \item Résoudre dans \( \mathbb{R} \) l'équation \( P(x) = 0 \).
    \item En déduire les solutions de l'équation \( P(x^2 - 2) = 0 \).
    \item Résoudre dans \( \mathbb{R} \) l'inéquation \( P(x) \leq 0 \).
    \item En déduire les solutions de l'inéquation \( P(3 - 2x) \leq 0 \).
    \item Résoudre dans \( \mathbb{R} \) l'inéquation \( \frac{2x^3 + x^2 - 7x - 6}{-x + 1} > 0 \).
\end{enumerate}

\section*{\underline{Exercice 3 :} 8 pts 	[\textit{Exercice 3 du devoir 1 déjà corrigé en classe} ]}

\begin{enumerate}
    \item Résoudre dans $\mathbb{R}$ les équations suivantes.
    \begin{enumerate}
        \item $\sqrt{-x^2 + 5x + 9} = \sqrt{x - 3}$
        \item $\sqrt{x^2 - x - 2} = x + 1$
    \end{enumerate}
    
    \item Résoudre dans $\mathbb{R}$ les inéquations suivantes.
    \begin{enumerate}
        \item $\sqrt{(x + 3)(x - 1)} \leq x - 3$
        \item $\sqrt{-5x + 1} \geq \sqrt{x^2 - 3x + 2}$
        \item $\sqrt{x^{2}-x-2} > x-1 $
        \item $x^4 + x^2 - 12 \leq 0$
    \end{enumerate}
    
    \item Résoudre dans $\mathbb{R}^2$ le système suivant
    \[
    \begin{cases}
        x^3 + y^3 = 9 \\
        \frac{x}{y} + \frac{y}{x} = \frac{5}{2}
    \end{cases}
    \]
    
    \item Résoudre dans $\mathbb{R}^3$ par la méthode du Pivot de Gauss le système
    \[
    \begin{cases}
        x - 2y + z = 6 \\
        -2x + y - z = -6 \\
        3x - y - 2z = -2
    \end{cases}
    \]
\end{enumerate}

\section*{\underline{Exercice 4 :} 1,5 pt [\textit{ Exercice d'application 1 du cours sur produit scalaire} ]} 
Soient $A(-2, -3)$, $B(1, 1)$, $C(-3, -1)$, $D(4, 2)$, $E(-4, -3)$ et $E(2, -1)$ dans un repère orthonormé.

En utilisant deux méthodes différentes (une méthode par triangle), montrer que le triangle $ABC$ est un triangle rectangle en $C$, mais que $FDE$ ne l'est pas en $E$.
\end{document}