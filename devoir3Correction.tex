\documentclass[12pt,a4paper]{article}
\usepackage{amsmath,amssymb,mathrsfs,tikz,times,pifont}
\usepackage{enumitem}
\usepackage{pgfplots}
\pgfplotsset{compat=1.15}
\usepackage{mathrsfs}
\usetikzlibrary{arrows}
\pagestyle{empty}
\newcommand\circitem[1]{%
\tikz[baseline=(char.base)]{
\node[circle,draw=gray, fill=red!55,
minimum size=1.2em,inner sep=0] (char) {#1};}}
\newcommand\boxitem[1]{%
\tikz[baseline=(char.base)]{
\node[fill=cyan,
minimum size=1.2em,inner sep=0] (char) {#1};}}
\setlist[enumerate,1]{label=\protect\circitem{\arabic*}}
\setlist[enumerate,2]{label=\protect\boxitem{\alph*}}
%%%::::::by chnini ameur :::::::%%%
\everymath{\displaystyle}
\usepackage[left=1cm,right=1cm,top=1cm,bottom=1.7cm]{geometry}
\usepackage{array,multirow}
\usepackage[most]{tcolorbox}
\usepackage{varwidth}
\tcbuselibrary{skins,hooks}
\usetikzlibrary{patterns}
%%%::::::by chnini ameur :::::::%%%
\newtcolorbox{exa}[2][]{enhanced,breakable,before skip=2mm,after skip=5mm,
colback=yellow!20!white,colframe=black!20!blue,boxrule=0.5mm,
attach boxed title to top left ={xshift=0.6cm,yshift*=1mm-\tcboxedtitleheight},
fonttitle=\bfseries,
title={#2},#1,
% varwidth boxed title*=-3cm,
boxed title style={frame code={
\path[fill=tcbcolback!30!black]
([yshift=-1mm,xshift=-1mm]frame.north west)
arc[start angle=0,end angle=180,radius=1mm]
([yshift=-1mm,xshift=1mm]frame.north east)
arc[start angle=180,end angle=0,radius=1mm];
\path[left color=tcbcolback!60!black,right color = tcbcolback!60!black,
middle color = tcbcolback!80!black]
([xshift=-2mm]frame.north west) -- ([xshift=2mm]frame.north east)
[rounded corners=1mm]-- ([xshift=1mm,yshift=-1mm]frame.north east)
-- (frame.south east) -- (frame.south west)
-- ([xshift=-1mm,yshift=-1mm]frame.north west)
[sharp corners]-- cycle;
},interior engine=empty,
},interior style={top color=yellow!5}}
%%%%%%%%%%%%%%%%%%%%%%%

\usepackage{fancyhdr}
\usepackage{eso-pic}         % Pour ajouter des éléments en arrière-plan
% Commande pour ajouter du texte en arrière-plan
\AddToShipoutPicture{
    \AtTextCenter{%
        \makebox[0pt]{\rotatebox{80}{\textcolor[gray]{0.7}{\fontsize{5cm}{5cm}\selectfont PGB}}}
    }
}
\usepackage{lastpage}
% D\'efinition de l'encadr\'e adaptatif avec fond jaune
\newtcolorbox{resultbox}{
    colback=red!30, % Fond rouge clair
    colframe=black, % Bordure noire fine
    sharp corners, % Coins nets
    boxrule=0.5pt, % Contour l\'eger
    boxsep=2pt, % Espacement interne
    left=5pt, right=5pt, top=2pt, bottom=2pt, % Marges internes
}
\fancyhf{}
\pagestyle{fancy}
\renewcommand{\footrulewidth}{1pt}
\renewcommand{\headrulewidth}{0pt}
\renewcommand{\footruleskip}{10pt}
\fancyfoot[R]{
\color{blue}\ding{45}\ \textbf{2025}
}
\fancyfoot[L]{
\color{blue}\ding{45}\ \textbf{Prof:M. BA}
}
\cfoot{\bf
\thepage /
\pageref{LastPage}}
\begin{document}
\renewcommand{\arraystretch}{1.5}
\renewcommand{\arrayrulewidth}{1.2pt}
\begin{tikzpicture}[overlay,remember picture]
    \node[draw=blue,line width=1.2pt,fill=purple,text=blue,inner sep=3mm,rounded corners,pattern=dots]at ([yshift=-2.5cm]current page.north) {\begingroup\setlength{\fboxsep}{0pt}\colorbox{white}{\begin{tabular}{|*1{>{\centering \arraybackslash}p{0.28\textwidth}} |*2{>{\centering \arraybackslash}p{0.2\textwidth}|} *1{>{\centering \arraybackslash}p{0.19\textwidth}|} }
                \hline
                \multicolumn{3}{|c|}{$\diamond$$\diamond$$\diamond$\ \textbf{Lycée de Dindéfélo}\ $\diamond$$\diamond$$\diamond$ } & \textbf{A.S. : 2024/2025}                                                                     \\ \hline
                \textbf{Matière: Mathématiques}                                                                                    & \textbf{Niveau : 1}\textbf{$^{er}$S2} & \textbf{Date: 19/03/2025} & \textbf{Durée : 4 heures} \\ \hline
                \multicolumn{4}{|c|}{\parbox[c]{10cm}{\begin{center}
                                                                  \textbf{{\Large\sffamily Devoir n$ ^{\circ} $ 1 Du 2$ ^\text{\bf nd} $ Semestre}}
                                                              \end{center}}}                                                                                                                               \\ \hline
            \end{tabular}}\endgroup};
\end{tikzpicture}
\vspace{3cm}

\section*{\underline{Exercice 4 :} 6 pts }
Dans le plan, on considère le triangle \( ABC \) tel que \( AB = 2 \), \( AC = 4\sqrt{2} \) et \( BC = 2\sqrt{5} \) (unité cm).
\( I \) est le milieu de \( [AB] \).

\begin{enumerate}
    \item
          \begin{enumerate}
              \item Calculons \( \overrightarrow{AB} \cdot \overrightarrow{AC} \). \hfill \textbf{(01 pt)}

                    \(
                    \begin{aligned}
                        \overrightarrow{BC}=\overrightarrow{BA}+\overrightarrow{AC} & \implies \overrightarrow{BC}^{2}=(\overrightarrow{BA}+\overrightarrow{AC})^{2}                                                                    \\
                                                                                    & \implies \overrightarrow{BC}^{2}=\overrightarrow{BA}^{2}+\overrightarrow{AC}^{2}+2\overrightarrow{BA}.\overrightarrow{AC}                         \\
                                                                                    & \implies \overrightarrow{BC}^{2}=\overrightarrow{BA}^{2}+\overrightarrow{AC}^{2}-2\overrightarrow{AB}.\overrightarrow{AC}                         \\
                                                                                    & \implies 2\overrightarrow{AB}.\overrightarrow{AC}=\overrightarrow{BA}^{2}+\overrightarrow{AC}^{2}-\overrightarrow{BC}^{2}                         \\
                                                                                    & \implies \overrightarrow{AB}.\overrightarrow{AC}=\frac{1}{2}\left( \overrightarrow{BA}^{2}+\overrightarrow{AC}^{2}-\overrightarrow{BC}^{2}\right) \\
                    \end{aligned}
                    \)

                    \begin{resultbox}
                        \[
                            \mathbf{\overrightarrow{AB}.\overrightarrow{AC}=\frac{1}{2}\left( \overrightarrow{BA}^{2}+\overrightarrow{AC}^{2}-\overrightarrow{BC}^{2}\right)}
                        \]
                    \end{resultbox}

                    \(
                    \begin{aligned}
                        \overrightarrow{AB}.\overrightarrow{AC} & =\frac{1}{2}\left( 2^{2}+(4\sqrt{2})^{2}-(2\sqrt{5})^{2}\right) \\
                        \overrightarrow{AB}.\overrightarrow{AC} & =\frac{1}{2}\left( 4+32-20\right)                               \\
                        \overrightarrow{AB}.\overrightarrow{AC} & =\frac{1}{2}\left( 36-20\right)                                 \\
                        \overrightarrow{AB}.\overrightarrow{AC} & =\frac{1}{2}\left( 16\right)                                    \\
                        \overrightarrow{AB}.\overrightarrow{AC} & =8
                    \end{aligned}
                    \)

                    \begin{resultbox}
                        \[
                            \mathbf{\overrightarrow{AB}.\overrightarrow{AC}=8}
                        \]
                    \end{resultbox}

              \item Déduisons-en \( \cos \widehat{BAC} \). \hfill \textbf{(0,5 pt)}

                    \(
                    \begin{aligned}
                        \overrightarrow{AB}.\overrightarrow{AC}= AB\times AC \times \cos \widehat{BAC} & \implies \cos \widehat{BAC}=\frac{\overrightarrow{AB}.\overrightarrow{AC}}{AB\times AC} \\
                                                                                                       & \implies \cos \widehat{BAC}=\frac{8}{2\times 4\sqrt{2}}                                 \\
                                                                                                       & \implies \cos \widehat{BAC}=\frac{1}{\sqrt{2}}                                          \\
                                                                                                       & \implies \cos \widehat{BAC}=\frac{\sqrt{2}}{2}                                          \\
                    \end{aligned}
                    \)

                    \begin{resultbox}
                        \[
                            \mathbf{cos \widehat{BAC}=\frac{\sqrt{2}}{2}}
                        \]
                    \end{resultbox}

              \item L'ensemble des points \( M \) du plan tels que \( \overrightarrow{BA} \cdot \overrightarrow{MC} = 0 \).

              \begin{resultbox}
                \[
                    \mathbf{\text{L'ensemble des points \( M \) du plan est la perpendiulaire à (AB) passant par C.} }
                \]
            \end{resultbox}
        
          \end{enumerate}

    \item Soit l'ensemble \( \mathcal{E} = \{ M \in \mathbb{P} \ / \ MA^2 + MB^2 = 6 \} \)
          \begin{enumerate}
              \item Montrons que \( MA^2 + MB^2 = 2MI^2 + 2 \). \hfill \textbf{(01 pt)}

                    \(
                    \begin{aligned}
                        \overrightarrow{MA}^2 + \overrightarrow{MB}^2 & = (\overrightarrow{MI}+\overrightarrow{IA})^2 + (\overrightarrow{MI}+\overrightarrow{IB})^2                                                                                  \\
                                                                      & = \overrightarrow{MI}^2+\overrightarrow{IA}^2+2\overrightarrow{MI}.\overrightarrow{IA}+ \overrightarrow{MI}^2+\overrightarrow{IB}^2+2\overrightarrow{MI}.\overrightarrow{IB} \\
                                                                      & = 2\overrightarrow{MI}^2+2\overrightarrow{MI}(\overrightarrow{IA}+\overrightarrow{IB})+\overrightarrow{IA}^2+\overrightarrow{IB}^2                                           \\
                                                                      & = 2\overrightarrow{MI}^2+\overrightarrow{IA}^2+\overrightarrow{IB}^2                                                                                                         \\
                                                                      & = 2\overrightarrow{MI}^2+\left( \frac{AB}{2}\right)^2+\left( \frac{AB}{2}\right)^{2}                                                                                         \\
                                                                      & = 2\overrightarrow{MI}^2+2\left( \frac{AB}{2}\right)^2                                                                                                                       \\
                                                                      & = 2\overrightarrow{MI}^2+\frac{AB}{2}^2                                                                                                                                      \\
                                                                      & = 2\overrightarrow{MI}^2+\frac{2}{2}^2                                                                                                                                       \\
                                                                      & = 2\overrightarrow{MI}^2+2 \textbf{ cqfd}                                                                                                                                    \\
                    \end{aligned}
                    \)
              \item Déterminons et construions l'ensemble \( \mathcal{E} \). \hfill \textbf{(0,5+0,5 pt)}

                    \(
                    \begin{aligned}
                        MA^2 + MB^2 = 6 & \implies 2\overrightarrow{MI}^2+2 = 6 \\
                                        & \implies 2\overrightarrow{MI}^2   = 4 \\
                                        & \implies \overrightarrow{MI}^2   = 2  \\
                    \end{aligned}
                    \)

                    L'ensemble \( \mathcal{E} \) est un cercle de centre \(I\) et de rayon \( \sqrt{2} \)

                    \[
                        \mathcal{E}=\{ C(I;\sqrt{2}) \}.
                    \]

                    \begin{resultbox}
                        \[
                            \mathbf{\mathcal{E} = \{C(I;\sqrt{2})\} }
                        \]
                    \end{resultbox}

                    \definecolor{wewdxt}{rgb}{0.43137254901960786,0.42745098039215684,0.45098039215686275}
                    \definecolor{zzttqq}{rgb}{0.6,0.2,0}
                    \definecolor{xdxdff}{rgb}{0.49019607843137253,0.49019607843137253,1}
                    \definecolor{ududff}{rgb}{0.30196078431372547,0.30196078431372547,1}
                    \begin{tikzpicture}[line cap=round,line join=round,>=triangle 45,x=1cm,y=1cm]
                        \clip(-7.5806713030775255,5.111029105064376) rectangle (5.023498620895737,12.441522187479684);
                        \fill[line width=2pt,color=zzttqq,fill=zzttqq,fill opacity=0.10000000149011612] (-4.559315779104064,9.226795625986128) -- (-2.5593157791040637,9.226795625986128) -- (-0.5593157791040646,5.226795625986126) -- cycle;
                        \draw [line width=2pt] (-4.559315779104064,9.226795625986128)-- (-2.5593157791040637,9.226795625986128);
                        \draw [line width=2pt,color=zzttqq] (-4.559315779104064,9.226795625986128)-- (-2.5593157791040637,9.226795625986128);
                        \draw [line width=2pt,color=zzttqq] (-2.5593157791040637,9.226795625986128)-- (-0.5593157791040646,5.226795625986126);
                        \draw [line width=2pt,color=zzttqq] (-0.5593157791040646,5.226795625986126)-- (-4.559315779104064,9.226795625986128);
                        \draw [line width=2pt] (-3.5593157791040637,9.226795625986128) circle (1.4142135623730951cm);
                        \begin{scriptsize}
                            \draw [fill=ududff] (-4.559315779104064,9.226795625986128) circle (2.5pt);
                            \draw[color=ududff] (-4.495446941791641,9.40154778349264) node {$A$};
                            \draw [fill=xdxdff] (-2.5593157791040637,9.226795625986128) circle (2.5pt);
                            \draw[color=xdxdff] (-2.496221555678389,9.40154778349264) node {$B$};
                            \draw [fill=wewdxt] (-0.5593157791040653,5.226795625986125) circle (2pt);
                            \draw[color=wewdxt] (-0.4969961695651367,5.3866424813392495) node {$C$};
                            \draw [fill=black] (-3.5593157791040637,9.226795625986128) circle (2pt);
                        \end{scriptsize}
                    \end{tikzpicture}
          \end{enumerate}

    \item Soit \( G \) le barycentre des points pondérés \( (A ; 2) \) ; \( (B ; -3) \) et \( \mathcal{F} = \{ M \in \mathbb{P} \ / \ 2MA^2 - 3MB^2 = 15 \} \)
          \begin{enumerate}
              \item Construisons \( G \) et calculer \( GA \) et \( GB \). \hfill \textbf{(0,5+0,5 pt)}

                    \( G \) le barycentre de \( (A ; 2) \) ; \( (B ; -3) \)

                    \(
                    \begin{aligned}
                        G =bar\{ (A ; 2) ; (B ; -3) \} & \implies \overrightarrow{AG} = 3 \overrightarrow{AB}  \\
                                                       & \implies \overrightarrow{BG} = -2 \overrightarrow{BA}
                    \end{aligned}
                    \)

                    \(
                    \begin{aligned}
                        \begin{cases}
                            \overrightarrow{AG} = 3 \overrightarrow{AB} \\
                            \overrightarrow{BG} = -2 \overrightarrow{BA}
                        \end{cases} & \iff
                        \begin{cases}
                            \| \overrightarrow{AG} \| =\| 3 \overrightarrow{AB}\| \\
                            \| \overrightarrow{BG} \|=\| -2 \overrightarrow{BA}\|
                        \end{cases} \\&\iff
                        \begin{cases}
                            AG = 3 AB \\
                            BG = 2 BA
                        \end{cases}                                          \\&\iff
                        \begin{cases}
                            GA = 6 \\
                            GB = 4
                        \end{cases}
                    \end{aligned}
                    \)

                    \begin{resultbox}
                        \[
                            \mathbf{ \begin{cases}
                                GA = 6 \\
                                GB = 4
                            \end{cases}}
                        \]
                    \end{resultbox}

              \item Montrons que \( 2MA^2 - 3MB^2 = -MG^2 + 24 \). \hfill \textbf{(01 pt)}

                    \(
                    \begin{aligned}
                        2MA^2 - 3MB^2 & = 2(\overrightarrow{MG}+\overrightarrow{GA})^2 - 3(\overrightarrow{MG}+\overrightarrow{GB})^2 \\
                                      & =2(MG^{2}+GA^{2}+2\overrightarrow{MG}.\overrightarrow{GA}) - 3(MG^{2}+GB^{2}+2\overrightarrow{MG}.\overrightarrow{GB})\\
                                      & =2MG^{2}+2GA^{2}+4\overrightarrow{MG}.\overrightarrow{GA} - 3MG^{2}-3GB^{2}-6\overrightarrow{MG}.\overrightarrow{GB}\\
                                      & =-MG^{2}+2GA^{2} -3GB^{2}+4\overrightarrow{MG}.\overrightarrow{GA}-6\overrightarrow{MG}.\overrightarrow{GB}\\
                                      & =-MG^{2}+2GA^{2} -3GB^{2}+2\overrightarrow{MG}(2\overrightarrow{GA}-3\overrightarrow{GB})\\
                                      & =-MG^{2}+2GA^{2} -3GB^{2}\\
                                      & =-MG^{2}+2(6)^{2} -3(4)^{2}\\
                                      & =-MG^{2}+2\times36 -3\times16\\
                                      & =-MG^{2}+72 -48\\
                                      & =-MG^{2}+24 \textbf{ cqfd}\\
                    \end{aligned}
                    \)
                    \item Déterminonns l’ensemble \( \mathcal{F} \). \hfill \textbf{(0,5 pt)}
                    
                    \( \mathcal{F} = \{ M \in \mathbb{P} \ / \ 2MA^2 - 3MB^2 = 15 \} \) d'après la question précédente, \( 2MA^2 - 3MB^2 = -MG^2 + 24 \)

                    \( 
                        \begin{aligned}
                            2MA^2 - 3MB^2 = 15 &\implies -MG^2 + 24 = 15\\
                                               &\implies -MG^2 + 24 = 15\\
                                               &\implies -MG^2 = 15 -24\\
                                               &\implies -MG^2 = -9\\
                                               &\implies MG^2 = 9\\
                                               &\implies MG = 3
                        \end{aligned}
                    \)

                    \begin{resultbox}
                        \[
                            \mathbf{\mathcal{F} = \{ \mathcal{C} (I,2)\}\text{   Donc l'ensemble des point \(M\) est un cercle de centre \(I\) et de rayon de 2 }}
                        \]
                    \end{resultbox}
                \end{enumerate}
\end{enumerate}
\end{document}