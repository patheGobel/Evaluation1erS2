\documentclass[a4paper,12pt]{article}
\usepackage{graphicx}
\usepackage[a4paper, top=0cm, bottom=2cm, left=2cm, right=2cm]{geometry} % Ajuste les marges
\usepackage{xcolor} % Pour ajouter des couleurs
\usepackage{hyperref} % Pour avoir des références colorées si nécessaire
\usepackage{eso-pic}         % Pour ajouter des éléments en arrière-plan

\usepackage[french]{babel}
\usepackage[T1]{fontenc}
\usepackage{mathrsfs}
\usepackage{amsmath}
\usepackage{amsfonts}
\usepackage{amssymb}
\usepackage{tkz-tab}

\usepackage{tikz}
\usetikzlibrary{arrows, shapes.geometric, fit}
\newcounter{correction} % Compteur pour les questions

% Définir la commande pour afficher une question numérotée
\newcommand{\question}{%
  \refstepcounter{correction}%
  \textbf{\textcolor{black}{Question \thecorrection :}} \ignorespaces
}
% Commande pour ajouter du texte en arrière-plan
\AddToShipoutPicture{
    \AtTextCenter{%
        \makebox[0pt]{\rotatebox{80}{\textcolor[gray]{0.9}{\fontsize{10cm}{10cm}\selectfont Pathé Gobel BA}}}
    }
}
\begin{document}
\hrule % Barre horizontale
% En-tête
\begin{center}
    \begin{tabular}{@{} p{5cm} p{5cm} p{5cm} @{}} % 3 colonnes avec largeurs fixées
        Lycée Dindéfélo & \quad\quad\quad Test 2 & 13 Novembre 2024 \\
    \end{tabular}
    \\[-0.01cm] % Ajuster l'espace vertical entre le tableau et la barre
    \hrule % Barre horizontale
\end{center}
\begin{center}
    \textbf{\Large Equations-Inéquations} \\[0.2cm]
    \textbf{\large Professeur : M. BA} \\[0.2cm]
    \textbf{Classe : $1^{er}$ S2} \\[0.2cm]
    \textbf{\small Durée : 10 minutes} \\[0.2cm]
    \textbf{\small Note :\quad\quad /5}
\end{center}

% Nom de l'élève
\textbf{\small Nom de l'élève :} \underline{\hspace{8cm}} \\[0.5cm]
\question\\ [0.5cm]\\ 
Résoudre dans $\mathbb{R}$ l'équation suivants
\[
\sqrt{2x + 1} = \sqrt{x + 3}
\]
\end{document}